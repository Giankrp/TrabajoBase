% Options for packages loaded elsewhere
\PassOptionsToPackage{unicode}{hyperref}
\PassOptionsToPackage{hyphens}{url}
%
\documentclass[
]{article}
\usepackage{amsmath,amssymb}
\usepackage{iftex}
\ifPDFTeX
  \usepackage[T1]{fontenc}
  \usepackage[utf8]{inputenc}
  \usepackage{textcomp} % provide euro and other symbols
\else % if luatex or xetex
  \usepackage{unicode-math} % this also loads fontspec
  \defaultfontfeatures{Scale=MatchLowercase}
  \defaultfontfeatures[\rmfamily]{Ligatures=TeX,Scale=1}
\fi
\usepackage{lmodern}
\ifPDFTeX\else
  % xetex/luatex font selection
\fi
% Use upquote if available, for straight quotes in verbatim environments
\IfFileExists{upquote.sty}{\usepackage{upquote}}{}
\IfFileExists{microtype.sty}{% use microtype if available
  \usepackage[]{microtype}
  \UseMicrotypeSet[protrusion]{basicmath} % disable protrusion for tt fonts
}{}
\makeatletter
\@ifundefined{KOMAClassName}{% if non-KOMA class
  \IfFileExists{parskip.sty}{%
    \usepackage{parskip}
  }{% else
    \setlength{\parindent}{0pt}
    \setlength{\parskip}{6pt plus 2pt minus 1pt}}
}{% if KOMA class
  \KOMAoptions{parskip=half}}
\makeatother
\usepackage{xcolor}
\setlength{\emergencystretch}{3em} % prevent overfull lines
\providecommand{\tightlist}{%
  \setlength{\itemsep}{0pt}\setlength{\parskip}{0pt}}
\setcounter{secnumdepth}{-\maxdimen} % remove section numbering
\ifLuaTeX
  \usepackage{selnolig}  % disable illegal ligatures
\fi
\ifPDFTeX
  \TeXXeTstate=1
  \newcommand{\RL}[1]{\beginR #1\endR}
  \newcommand{\LR}[1]{\beginL #1\endL}
  \newenvironment{RTL}{\beginR}{\endR}
  \newenvironment{LTR}{\beginL}{\endL}
\fi
\usepackage{bookmark}
\IfFileExists{xurl.sty}{\usepackage{xurl}}{} % add URL line breaks if available
\urlstyle{same}
\hypersetup{
  pdftitle={Introducción al proyecto},
  hidelinks,
  pdfcreator={LaTeX via pandoc}}

\title{Introducción al proyecto}
\author{}
\date{}

\begin{document}
\maketitle

\#Trabajobase\\
El presente proyecto se desarrolla en el contexto del módulo de
Administración de Bases de Datos (DAM). El objetivo es diseñar e
implementar una base de datos en MySQL para gestionar información de
manera eficiente. Después de analizar posibles escenarios, se decidió
que el tema central de este proyecto será una base de datos para la
gestión de inventarios.

\section{Elección del tema: Base de datos para
inventario}\label{elecciuxf3n-del-tema-base-de-datos-para-inventario}

La elección de este tema surge a partir de la identificación de
problemas comunes en el control de inventarios. Durante la evaluación
inicial, se consideraron distintos tipos de bases de datos que podrían
implementarse, como:

\begin{itemize}
\tightlist
\item
  Gestión de usuarios
\item
  Sistema de ventas
\item
  Gestión de inventario\\
  Finalmente, se optó por desarrollar una base de datos para
  \textbf{gestionar inventarios} debido a su relevancia en el ámbito
  empresarial, su complejidad moderada (ideal para aprendizaje) y su
  amplio uso en distintos sectores. Algunas de las razones que sustentan
  esta elección son:
\end{itemize}

\begin{enumerate}
\tightlist
\item
  \textbf{Relevancia Práctica:} La gestión de inventarios es crucial
  para el funcionamiento de cualquier negocio, ya que permite un
  seguimiento eficiente de los productos, optimización de recursos y
  prevención de pérdidas.
\item
  \textbf{Cobertura de Funcionalidades:} Este tema involucra múltiples
  aspectos de bases de datos, como la relación entre entidades,
  consultas complejas y manejo de restricciones.
\item
  \textbf{Escalabilidad:} La base de datos podría extenderse en el
  futuro para incluir funcionalidades como reportes avanzados o
  integración con sistemas de ventas.
\end{enumerate}

Check

Hecho

\end{document}
